\documentclass[12pt,a4paper,openright,twoside]{book}
\usepackage[utf8]{inputenc}
\usepackage{disi-thesis}
\usepackage{code-lstlistings}
\usepackage{notes}
\usepackage{shortcuts}
\usepackage{acronym}


\school{\unibo}
\programme{Corso di Laurea in Ingegneria e Scienze Informatiche}
\title{ClientShield: Implementazione di un Servizio Windows per la Sicurezza DNS}
\author{Federico Diotallevi}
\date{\today}
\subject{Programmazione ad Oggetti}
\supervisor{Mirko Viroli}
%\cosupervisor{Nicolas Farabegoli}
\session{IV}
\academicyear{2023-2024}

% Definition of acronyms
\acrodef{DoH}{Dns over Https}
\acrodef{DNS}{Domain Name System}


\mainlinespacing{1.241} % line spacing in mainmatter, comment to default (1)

\begin{document}

\frontmatter\frontispiece

\begin{abstract}

Lo sviluppo della tecnologia cresce sempre più velocemente e, con essa, anche i rischi che gli utenti corrono semplicemente navigando in rete.
Questa tesi si inserisce nell'ambito dell'internet filtering, cioè la possibiità di filtrare il traffico in rete negando l'accesso a siti potenzialmente pericolosi.

L'obiettivo di questo progetto è lo sviluppo di un software che sia facilmente installabile su un dispositivo Windows e filtri tutte le richieste web effettuate.
Si vuole offrire la possibilità ad un un amministratore di rete di attivare e disattivare la protezione del dispositivo da remoto, decidendo eventualmente anche quali categorie di siti bloccare.

Per lo sviluppo è stata fondamentale la collaborazione con FlashStart, azienda di riferimento nel mercato italiano per quanto riguarda il DNS Filtering.

Il software è stato sviluppato in contesto aziendale come prototipo e sfrutta ampiamente le soluzioni di filtraggio aziendale.
Ciò ha facilitato la gestione delle categorie e dell'effettivo filtraggio DNS, permettendo al progetto di focalizzarsi sulla cattura e redirezione del traffico internet del dispositivo verso i server gestiti da FlashStart.
\end{abstract}

\begin{dedication}
Optional dedication.
\end{dedication}

%----------------------------------------------------------------------------------------
\tableofcontents   
\listoffigures     % (optional) comment if empty
\lstlistoflistings % (optional) comment if empty
%----------------------------------------------------------------------------------------

\mainmatter

%----------------------------------------------------------------------------------------
\chapter{Introduzione}
\label{chap:introduction}
%----------------------------------------------------------------------------------------

Navigare in rete è senz'altro una delle attività più diffuse al giorno d'oggi in tutto il mondo.
La quantità di informazioni a cui si può accedere è pressoché infinita e soprattutto a disposizione di chiunque.

Nonostante si possa essere portati a credere che le nuove generazioni siano sempre meno soggette  ad attacchi informatici, essendo sempre più abituati ad utilizzare strumenti tecnologici, la realtà si dimostra molto diversa.
Secondo \cite{unipol2022}, infatti, le violazioni digitali colpiscono soprattutto la Generazione Z.
Non solo, l'articolo evidenzia che gli italiani ch ehanno subito violazioni digitali sono circa 10 milioni e che il rischio ha una frequenza maggiore tra chi utilizza Social Network molto frequentemente.




\paragraph{Structure of the Thesis}

\chapter{State of the art}

I suggest referencing stuff as follows: \cref{fig:random-image} or \Cref{fig:random-image}

\begin{figure}
    \centering
    \includegraphics[width=.8\linewidth]{figures/random-image.pdf}
    \caption{Some random image}
    \label{fig:random-image}
\end{figure}

\section{Some cool topic}

\chapter{Contribution}

You may also put some code snippet (which is NOT float by default), eg: \cref{lst:random-code}.


\lstinputlisting[float,language=Java,label={lst:random-code}]{listings/HelloWorld.java}

\section{Fancy formulas here}

%----------------------------------------------------------------------------------------
% BIBLIOGRAPHY
%----------------------------------------------------------------------------------------

\backmatter

\bibliographystyle{alpha}
\bibliography{bibliography}

\begin{acknowledgements} % this is optional
Optional. Max 1 page.
\end{acknowledgements}

\end{document}