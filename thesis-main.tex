% !TeX spellcheck = it_IT
% !TeX root = thesis-main.tex
\documentclass[12pt,a4paper,openright,twoside]{book}
\usepackage[utf8]{inputenc}
\usepackage[italian]{babel}
\usepackage[hyperfootnotes=false]{hyperref}
\usepackage{disi-thesis}
\usepackage[acronym,toc]{glossaries}
\usepackage{code-lstlistings}
\usepackage{notes}
\usepackage{shortcuts}
\usepackage{float}
\usepackage{pgfplots}

\newcommand{\acknowledgementsname}{Riconoscimenti}
\renewcommand{\chaptername}{Capitolo} % Per i capitoli
\renewcommand{\bibname}{Bibliografia} % Per la bibliografia
\renewcommand{\contentsname}{Indice}  % Per il sommario
\renewcommand{\listfigurename}{Elenco delle figure}
\renewcommand{\listtablename}{Elenco delle tabelle}

\glsdisablehyper


\school{\unibo}
\programme{Corso di Laurea in Ingegneria e Scienze Informatiche}
\title{ClientShield: Implementazione di un Servizio Windows per la Sicurezza DNS}
\author{Federico Diotallevi}
\date{\today}
\subject{Programmazione ad Oggetti}
\supervisor{Prof.\ Viroli Mirko}
%\cosupervisor{Nicolas Farabegoli}
\session{IV}
\academicyear{2023\ -\ 2024}

\newacronym{DoH}{DoH}{DNS over HTTPS}
\newacronym{DNS}{DNS}{Domain Name System}
\newacronym{AI}{AI}{Intelligenza Artificiale}
\newacronym{PoC}{PoC}{Proof of Concept}
\newacronym{AGID}{AGID}{Agenzia per l'Italia Digitale}
\newacronym{VPN}{VPN}{Virtual Private Network}
\newacronym{HTTPS}{HTTPS}{HTTP Secure}
\newacronym{ISP}{ISP}{Internet Service Provider}


\mainlinespacing{1.241} % line spacing in mainmatter, comment to default (1)

\begin{document}

\frontmatter\frontispiece\

\renewcommand{\abstractname}{Sommario}
\begin{abstract}

Lo sviluppo della tecnologia cresce sempre più velocemente e, con essa, anche i rischi che gli utenti corrono semplicemente navigando in rete.
Questa tesi si inserisce nell'ambito dell'internet filtering, cioè la possibilità di filtrare il traffico in rete negando l'accesso a siti potenzialmente pericolosi.

L'obiettivo di questo progetto è lo sviluppo di un software che sia facilmente installabile su un dispositivo Windows e filtri tutte le richieste web effettuate.
Si vuole offrire la possibilità ad un un amministratore di rete di attivare e disattivare la protezione del dispositivo da remoto, decidendo eventualmente anche quali categorie di siti bloccare.

Per lo sviluppo è stata fondamentale la collaborazione con FlashStart, azienda di riferimento nel mercato italiano per quanto riguarda il filtraggio \gls{DNS}.

Il software è stato sviluppato in contesto aziendale come prototipo e sfrutta ampiamente le soluzioni di filtraggio aziendali.
Ciò ha facilitato la gestione delle categorie e dell'effettivo filtraggio \gls{DNS}, permettendo al progetto di focalizzarsi sulla cattura e redirezione del traffico internet del dispositivo verso i server gestiti da FlashStart.
\end{abstract}

\begin{dedication}
Optional dedication.
\end{dedication}

%----------------------------------------------------------------------------------------
\tableofcontents   
%\listoffigures     % (optional) comment if empty
%\lstlistoflistings\ % (optional) comment if empty
%----------------------------------------------------------------------------------------

\mainmatter\

%----------------------------------------------------------------------------------------
\chapterWithoutNumber{Introduzione}
\phantomsection % to makr label work
\label{chap:introduzione}
%----------------------------------------------------------------------------------------

Navigare in rete è senz'altro una delle attività più diffuse al giorno d'oggi in tutto il mondo.
La quantità di informazioni a cui si può accedere è pressoché infinita e, soprattutto, a disposizione di chiunque.

La digitalizzazione delle risorse è ormai diventata essenziale per qualsiasi realtà, dal settore pubblico a quello privato.
Tuttavia, tale pratica espone inevitabilmente al rischio che documenti,
dati e informazioni sensibili diventino obiettivo di attacchi informatici.
A conferma di ciò, il recente rapporto \cite{clusit2024-sicurezza} evidenzia come gli attacchi informatici gravi a livello globale siano passati,
in media, da 4.5 a 9 al giorno in soli cinque anni.
Secondo il rapporto, gli incidenti critici sono aumentati dal 47\% all'81\% del totale e,
in Italia, il 58\% degli attacchi è costituito da malware, phishing e social engineering.

Con il recente e rapidissimo sviluppo dell' \gls{AI} il campo dell'Internet Filtering ha subito un vera e propria rivoluzione.
I nuovi modelli di Machine Learning sono capaci di analizzare e identificare con precisione siti malevoli,
grazie anche ai giganteschi volumi di dati di cui dispongono.
Questi sistemi riescono a riconoscere tecniche di phishing e malware avanzate, anche quando progettate per eludere i metodi di rilevamento tradizionali.

Questa tesi si colloca in questo contesto e si propone l'obiettivo di sviluppare un software prototipale per il filtraggio internet sul computer di installazione.
Il lavoro illustrerà dettagliatamente tutte le fasi di analisi, progettazione e sviluppo del progetto.

Data la necessità di interazione con componenti specifici del sistema operativo,
il software è stato sviluppato esclusivamente per Windows, con l'obiettivo di creare un servizio di sistema.

\paragraph{Struttura della Tesi}

La tesi si articola in 4 capitoli principali, che descrivono le diverse fasi della realizzazione del progetto, in particolare:

\begin{itemize}

    \item \textbf{Capitolo 1 (Contesto)}: in questo capitolo si tratta in maniera approfondita il problema affrontato dalla tesi.
    Viene analizzato il software attualmente presente in FlashStart e le motivazioni che inducono alla necessità di un nuovo sviluppo.
    
    \item \textbf{Capitolo 2 (Requisiti)}: in questo capitolo vengono affrontati casi d'uso e requisiti, analizzando il funzionamento e la percezione del progetto dal punto di vista dell'utente.
    
    \item \textbf{Capitolo 3 (Tecnologie)}:
    in questo capitolo vengono riportate le diverse tecnologie utilizzate e le scelte che hanno portato al loro impiego nel progetto.
    
    \item \textbf{Capitolo 4 (Analisi e progettazione)}:
    in questo capitolo viene descritta la fase di analisi e progettazione del sistema,
    illustrandone in particolare l'architettura e i pattern utilizzati, evidenziando in che modo il software ne tragga guadagno.

\end{itemize}


\chapter{Contesto}
\label{chap:contesto}

\section{Internet Filtering}

Il caso di studio oggetto di questa tesi è un software prototipale, che si inserisce nell'ambito dell'Internet Filtering.
Per Internet Filtering si intende un sistema di monitorazione, controllo e limitazione dell'accesso alle risorse online, basato su criteri predefiniti.

Una tecnologia di questo tipo risulta particolarmente utile ad aziende, scuole ed enti pubblici.
Tramite questa pratica è infatti possibile effettuare controlli sul contenuto delle pagine web, limitando o impedendo l'accesso a siti contenenti malware o, eventualmente, contenuti indesiderati (ad esempio siti di scommesse, pornografia, droghe...).
Risulta evidente fin da subito che per garantire un filtraggio di contenuti preciso ed efficace è necessaria un'attenta analisi delle pagine web, per evitare di correre il rischio di impedirne immotivatamente l'accesso.

Questa tesi è svolta come progetto in collaborazione con FlashStart, leader italiana per quanto riguarda il filtraggio \gls{DNS}.
Un server \gls{DNS} è un sistema che permette di tradurre nomi di domini leggibili da un uomo, in indirizzi interpretabili da dei computer per connettersi tra loro.
Il filtraggio \gls{DNS} si basa proprio su questo principio: quando si accede ad un sito web attraverso un browser qualsiasi, il computer interrogherà un server \gls{DNS} per ottenere l'indirizzo attraverso cui sia possibile accedere al sito.
FlashStart offre un servizio \gls{DNS} configurabile, attraverso cui è possibile definire delle categorie di siti da bloccare.
I server dell'azienda aggiornano continuamente la lista delle categorie bloccate, facendo largo uso di \gls{AI} per garantire rapido aggiornamento (in risposta alla nascita di nuovi siti) ed alta precisione.
Quando un utente, utilizzando un computer protetto dal servizio di FlashStart, tenta di accedere ad un sito appartenente ad una delle categorie bloccate, il \gls{DNS} risponderà alla richiesta di risoluzione del nome con l'indirizzo di una pagina indicante i motivi del blocco, al posto di quella richiesta.

\section{Obiettivi del Progetto}

Il filtraggio \gls{DNS} può essere implementato in diverse modalità, di seguito le principali:

\begin{itemize}
	\item \textbf{A livello di rete:} applicato direttamente dall'\gls{ISP}\footref{foot:isp-footnote}, blocca l'accesso a domini malevoli prima che le richieste raggiungano la rete locale.  
	\newline \textit{Esempio: un'azienda utilizza un servizio \gls{DNS} filtrato fornito dal proprio \gls{ISP} per impedire l'accesso a siti malevoli su tutta la rete aziendale.}
	
	\item \textbf{A livello di router:} il router funge da punto di controllo, utilizzando un server DNS con filtraggio per impedire l'accesso a contenuti non autorizzati. Tutti i dispositivi connessi alla rete collegata al router beneficiano automaticamente della protezione.
	\newline \textit{Esempio: un istituto scolastico configura il proprio router con un DNS appositamente configurato per impedire l'accesso a siti per adulti e social network bloccandoli per gli studenti.}
	
	\item \textbf{A livello di endpoint:} il filtraggio DNS è applicato direttamente sui dispositivi tramite configurazione manuale o software specifici, garantendo protezione anche fuori dalla rete aziendale o domestica.  
	\newline \textit{Esempio: un dipendente in che lavora da remoto utilizza un client DNS sicuro sul proprio laptop per proteggersi dalle minacce.}
\end{itemize}

\footnote{\label{foot:isp-footnote}
	Un \gls{ISP} è un fornitore che offre connettività alla rete, consentendo ai suoi utenti di accedere a Internet. Gli ISP possono anche fornire servizi aggiuntivi, come, appunto, sicurezza informatica e filtraggio.}

Questo progetto prende in esame la terza tipologia delle suddette modalità di protezione \gls{DNS}, ossia il filtraggio a livello di endpoint.
Nel periodo tra il 2019 e 2021 le aziende italiane, così come quelle di tutto il mondo, hanno dovuto affrontare la necessità dell'impiego del lavoro remoto.
Se da una parte tale pratica rappresenta un'opportunità per i lavoratori, che ne guadagnano in flessibilità, dall'altra espone l'azienda a nuovi rischi.
A dimostrazione di ciò, proprio nel 2020 è stato emanato dall'\gls{AGID} il vademecum \cite{AgID2020}, cioè una serie di undici raccomandazioni che offrono linee guida per garantire la sicurezza del lavoro da remoto.

In effetti, un dispositivo che esce dall'azienda, ambiente potenzialmente controllato e protetto, diventa un rischio.
Un dipendente potrebbe involontariamente navigare in siti di phishing e malware o connettersi a reti Wi-Fi infette, esponendo l'azienda al rischio di furti o compromissione di dati.
Obiettivo fondante di questa tesi è lo sviluppo di un software facilmente installabile su un computer Windows, che consenta una protezione efficace e completa ai dispositivi desktop che escono dalla rete aziendale.
Tale progetto dovrà filtrare la connessione dell'endpoint remoto, bloccando l'accesso a siti pericolosi, come virus e frodi, ma anche a contenuti indesiderati e distrazioni sul lavoro, come social networks o videogiochi.

\section{Contesto Aziendale}

FlashStart offre già un software di protezione da malware e contenuti indesiderati dedicato ad endpoint remoti: ClientShield.
Tale prodotto rappresenta un'estensione del filtro aziendale, fornendo lo stesso tipo di protezione, configurabile attraverso la medesima piattaforma cloud da parte di un amministratore di rete.

Allo stato dell'arte il prodotto utilizza una \gls{VPN}, per stabilire una comunicazione sicura con i server di FlashStart.
Ciò garantisce che il dispositivo possa inviare e ricevere non soltanto le informazioni sul prodotto, ma anche richieste \gls{DNS}, generalmente trasmesse in chiaro, in modo totalmente criptato.

L'esigenza di aggiornamento del software deriva dalla volontà di FlashStart di integrare il protocollo \gls{DoH}, descritto in \cite{RFC8484}, nei suoi prodotti.
Rispetto ad una classica \gls{VPN}, \gls{DoH} sfrutta la crittografia asimmetrica, implementata dal protocollo \gls{HTTPS}, per garantire autenticazione, confidenzialità e integrità delle richieste \gls{DNS}.

Il vantaggio che FlashStart trae da entrambi i protocolli è essenzialmente il medesimo: proteggere le comunicazioni \gls{DNS} tramite la crittografia, evitando la possibilità di attacchi dovuti alla trasparenza delle informazioni.
La differenza principale tra i due approcci è che una \gls{VPN} cripta il traffico di rete nella sua interezza, mascherando anche l'indirizzo IP del dispositivo, mentre \gls{DoH} si concentra esclusivamente sulla protezione delle richieste \gls{DNS}.

Per un'azienda come FlashStart, specializzata in filtraggio \gls{DNS}, mascherare l'indirizzo IP di un endpoint non è una funzionalità totalmente rilevante per l'erogazione del suo servizio.
L'adozione del protocollo \gls{DoH} consente quindi di ottenere la riservatezza necessaria per la comunicazione con il \gls{DNS}, eliminando la necessità di un'infrastruttura \gls{VPN} dedicata e riducendo l'overhead complessivo del sistema.

La novità principale del software prototipale illustrato in questa tesi, è dunque quella di fornire una soluzione per la transizione dalla tecnologia \gls{VPN}, all'adozione del protocollo \gls{DoH}.
Di fondamentale importanza è inoltre l'analisi e la valutazione di soluzioni architetturali più adatte garantire la naturale evoluzione del software, facilitando l'introduzione di nuove funzionalità.

\chapter{Requisiti}

%----------------------------------------------------------------------------------------
% BIBLIOGRAPHY
%----------------------------------------------------------------------------------------

\backmatter\

\bibliographystyle{alpha}
\bibliography{bibliography}

\begin{acknowledgements} % this is optional
Optional. Max 1 page.
\end{acknowledgements}

\end{document}