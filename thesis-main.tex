% !TeX root = thesis-main.tex
\documentclass[12pt,a4paper,openright,twoside]{book}
\usepackage[hyperfootnotes=false]{hyperref}
\usepackage[italian,english]{babel}
\usepackage[utf8]{inputenc}
\usepackage{disi-thesis}
\usepackage{code-lstlistings}
\usepackage{notes}
\usepackage{shortcuts}
\usepackage{acronym}
\usepackage{float}
\usepackage{pgfplots}


\school{\unibo}
\programme{Corso di Laurea in Ingegneria e Scienze Informatiche}
\title{ClientShield: Implementazione di un Servizio Windows per la Sicurezza DNS}
\author{Federico Diotallevi}
\date{\today}
\subject{Programmazione ad Oggetti}
\supervisor{Prof.\ Viroli Mirko}
%\cosupervisor{Nicolas Farabegoli}
\session{IV}
\academicyear{2023\ -\ 2024}

% Definition of acronyms
\acrodef{DoH}{Dns over Https}
\acrodef{DNS}{Domain Name System}
\acrodef{AI}{Intelligenza Artificiale}



\mainlinespacing{1.241} % line spacing in mainmatter, comment to default (1)

\begin{document}

\frontmatter\frontispiece\

\renewcommand{\abstractname}{Sommario}
\begin{abstract}

Lo sviluppo della tecnologia cresce sempre più velocemente e, con essa, anche i rischi che gli utenti corrono semplicemente navigando in rete.
Questa tesi si inserisce nell'ambito dell'internet filtering, cioè la possibilità di filtrare il traffico in rete negando l'accesso a siti potenzialmente pericolosi.

L'obiettivo di questo progetto è lo sviluppo di un software che sia facilmente installabile su un dispositivo Windows e filtri tutte le richieste web effettuate.
Si vuole offrire la possibilità ad un un amministratore di rete di attivare e disattivare la protezione del dispositivo da remoto, decidendo eventualmente anche quali categorie di siti bloccare.

Per lo sviluppo è stata fondamentale la collaborazione con FlashStart, azienda di riferimento nel mercato italiano per quanto riguarda il filtraggio DNS.

Il software è stato sviluppato in contesto aziendale come prototipo e sfrutta ampiamente le soluzioni di filtraggio aziendali.
Ciò ha facilitato la gestione delle categorie e dell'effettivo filtraggio DNS, permettendo al progetto di focalizzarsi sulla cattura e redirezione del traffico internet del dispositivo verso i server gestiti da FlashStart.
\end{abstract}

\begin{dedication}
Optional dedication.
\end{dedication}

%----------------------------------------------------------------------------------------
\tableofcontents   
%\listoffigures     % (optional) comment if empty
%\lstlistoflistings\ % (optional) comment if empty
%----------------------------------------------------------------------------------------

\mainmatter\

%----------------------------------------------------------------------------------------
\chapterWithoutNumber{Introduzione}\label{chap:introduzione}



%----------------------------------------------------------------------------------------

Navigare in rete è senz'altro una delle attività più diffuse al giorno d'oggi in tutto il mondo.
La quantità di informazioni a cui si può accedere è pressoché infinita e, soprattutto, a disposizione di chiunque.

La digitalizzazione delle risorse è ormai diventata essenziale per qualsiasi realtà, dal settore pubblico a quello privato.
Tuttavia, tale pratica espone inevitabilmente al rischio che documenti,
dati e informazioni sensibili diventino obiettivo di attacchi informatici.
A conferma di ciò, il recente rapporto \cite{clusit2024-sicurezza} evidenzia come gli attacchi informatici gravi a livello globale siano passati,
in media, da 4.5 a 9 al giorno in soli cinque anni.
Secondo il rapporto, gli incidenti critici sono aumentati dal 47\% all'81\% del totale e,
in Italia, il 58\% degli attacchi è costituito da malware, phishing e social engineering.

Con il recente e rapidissimo sviluppo delle \ac{AI} il campo dell'Internet Filtering ha subito un vera e propria rivoluzione.
I nuovi modelli di Machine Learning sono capaci di analizzare e identificare con precisione siti malevoli,
grazie anche ai giganteschi volumi di dati di cui dispongono.
Questi sistemi riescono a riconoscere tecniche di phishing e malware avanzate, anche quando progettate per eludere i metodi di rilevamento tradizionali.

Questa tesi si colloca in questo contesto e si propone l'obiettivo di sviluppare un software prototipale per il filtraggio internet sul computer di installazione.
Il lavoro illustrerà dettagliatamente tutte le fasi di analisi, progettazione e sviluppo del progetto.

Data la necessità di interazione con componenti specifici del sistema operativo,
il software è stato sviluppato esclusivamente per Windows, con l'obiettivo di creare un servizio di sistema.

\paragraph{Struttura della Tesi}

La tesi si articola in 4 capitoli principali, che descrivono le diverse fasi della realizzazione del progetto, in particolare:

\begin{itemize}

    \item \textbf{Capitolo 1 (Contesto)}: in questo capitolo si tratta in maniera approfondita il contesto aziendale e il problema affrontato dalla tesi.
    Viene analizzato il software attualmente presente in FlashStart e le motivazioni che inducono alla necessità di un nuovo sviluppo.
    
    \item \textbf{Capitolo 2 (Requisiti)}: in questo capitolo vengono affrontati casi d'uso e requisiti, analizzando il funzionamento e la percezione del progetto dal punto di vista dell'utente.
    
    \item \textbf{Capitolo 3 (Tecnologie)}:
    in questo capitolo vengono riportate le diverse tecnologie utilizzate e le scelte che hanno portato al loro impiego nel progetto.
    
    \item \textbf{Capitolo 4 (Analisi e progettazione)}:
    in questo capitolo viene descritta la fase di analisi e progettazione del sistema,
    illustrandone in particolare l'architettura e i pattern utilizzati, evidenziando in che modo il software ne tragga guadagno.

\end{itemize}


\chapter{Contesto}

\begin{figure}
    \centering
    \includegraphics[width=.8\linewidth]{figures/random-image.pdf}
    \caption{Some random image}
\end{figure}\label{fig:random-image}

\section{Some cool topic}

\chapter{Contribution}

You may also put some code snippet (which is NOT float by default), eg: \cref{lst:random-code}.


\lstinputlisting[float,language=Java,label={lst:random-code}]{listings/HelloWorld.java}

\section{Fancy formulas here}

%----------------------------------------------------------------------------------------
% BIBLIOGRAPHY
%----------------------------------------------------------------------------------------

\backmatter\

\bibliographystyle{alpha}
\bibliography{bibliography}

\begin{acknowledgements} % this is optional
Optional. Max 1 page.
\end{acknowledgements}

\end{document}