\documentclass[12pt,a4paper,openright,twoside]{book}
\usepackage[utf8]{inputenc}
\usepackage{disi-thesis}
\usepackage{code-lstlistings}
\usepackage{notes}
\usepackage{shortcuts}
\usepackage{acronym}


\school{\unibo}
\programme{Corso di Laurea in Ingegneria e Scienze Informatiche}
\title{ClientShield: Implementazione di un Servizio Windows per la Sicurezza DNS}
\author{Federico Diotallevi}
\date{\today}
\subject{Programmazione ad Oggetti}
\supervisor{Mirko Viroli}
%\cosupervisor{Nicolas Farabegoli}
\session{IV}
\academicyear{2023-2024}

% Definition of acronyms
\acrodef{DoH}{Dns over Https}
\acrodef{DNS}{Domain Name System}


\mainlinespacing{1.241} % line spacing in mainmatter, comment to default (1)

\begin{document}

\frontmatter\frontispiece

\begin{abstract}

Lo sviluppo della tecnologia cresce sempre più velocemente e, con essa, anche le minacce che gli utenti in rete corrono sempre più.
Questa tesi si inserisce nell'ambito dell'internet filtering, cioè la possibiità di filtrare il traffico in rete negando l'accesso a siti potenzialmente pericolosi.

L'obiettivo di questo progetto è lo sviluppo di un software che filtri tutte le richieste web effettuate da un dispositivo, in modo che un amministratore di rete possa eventualmente decidere delle categorie di siti di cui bloccare l'accesso.
Per lo sviluppo è stata fondamentale la collaborazione con FlashStart, azienda di riferimento nel mercato italiano per quanto riguarda il DNS Filtering.

Il software è stato sviluppato in contesto aziendale come prototipo e sfrutta ampiamente le soluzioni di filtraggio aziendale.
Ciò ha facilita la gestione delle categorie e dell'effettivo filtraggio dns, permettendo al progetto di focalizzarsi sulla cattura e redirezione del traffico internet del dispositivo verso i server DNS di FlashStart.
\end{abstract}

\begin{dedication}
Optional dedication.
\end{dedication}

%----------------------------------------------------------------------------------------
\tableofcontents   
\listoffigures     % (optional) comment if empty
\lstlistoflistings % (optional) comment if empty
%----------------------------------------------------------------------------------------

\mainmatter

%----------------------------------------------------------------------------------------
\chapter{Introduction}
\label{chap:introduction}
%----------------------------------------------------------------------------------------

Write your intro here.
\sidenote{Add sidenotes in this way. They are named after the author of the thesis}

You can use acronyms that your defined previously,
such as \ac{IoT}.
%
If you use acronyms twice,
they will be written in full only once
(indeed, you can mention the \ac{IoT} now without it being fully explained).
%
In some cases, you may need a plural form of the acronym.
%
For instance,
that you are discussing \acp{vm},
you may need both \ac{vm} and \acp{vm}.

\paragraph{Structure of the Thesis}

\note{At the end, describe the structure of the paper}

\chapter{State of the art}

I suggest referencing stuff as follows: \cref{fig:random-image} or \Cref{fig:random-image}

\begin{figure}
    \centering
    \includegraphics[width=.8\linewidth]{figures/random-image.pdf}
    \caption{Some random image}
    \label{fig:random-image}
\end{figure}

\section{Some cool topic}

\chapter{Contribution}

You may also put some code snippet (which is NOT float by default), eg: \cref{lst:random-code}.

\lstinputlisting[float,language=Java,label={lst:random-code}]{listings/HelloWorld.java}

\section{Fancy formulas here}

%----------------------------------------------------------------------------------------
% BIBLIOGRAPHY
%----------------------------------------------------------------------------------------

\backmatter

\nocite{*} % Remove this as soon as you have the first citation

\bibliographystyle{alpha}
\bibliography{bibliography}

\begin{acknowledgements} % this is optional
Optional. Max 1 page.
\end{acknowledgements}

\end{document}
